\documentclass[11pt, letterpaper]{article}
\usepackage[utf8]{inputenc}

\title{Prueba5}
\author{Rolando Rivas}

\date{21/04/2023}

\begin{document}
\maketitle

\section{Prueba 5. Maestra experimentada, directora de carrera}
\subsection{Caso:}
 
{\large Una estudiante de 3ero de primaria llamada Mercedes Brewing acaba de tener un episodio de TDA, hubo un berrinche en el salón y la niña se cerró a las pláticas. Sin embargo, tu, como maestra recién entrada a la institución, no sabes que fue el estímulo que lo detonó. ¿En dónde buscarías aquellos estímulos que pudieron haberle detonado estas emociones y este episodio emocional?}

Áreas por oportunidad: 
\begin{itemize}
    \item Agregar una barra de busqueda 
    \item La barra del menu no se aprecia en pantallas chicas. Se tiene que hacer scroll para abajo
\end{itemize}
Métricos observados:
\begin{itemize}
    \item Tuvo ciertas dificultades para ver iconos chicos (gente de edad media tienen mas posibilidades de tener hipermetropia) 
    \item No se tuvo exito 
\end{itemize}
Fortalezas:
\begin{itemize}
    \item Navegar es sencillo, no hubo erratas en al seleccionar opciones
\end{itemize}
\end{document}