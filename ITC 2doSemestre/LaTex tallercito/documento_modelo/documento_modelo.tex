%%-----------------------------%%
%%----Comienza el preámbulo----%%
%%-----------------------------%%

%Tipo de documento
\documentclass[]{article}

%Paquetes
\usepackage[spanish,es-tabla]{babel} %Uso del idioma español
\usepackage[a4paper, total={7in, 9in}]{geometry} %Cambiar el tamaño del área impresa
\usepackage{lipsum} %Genera el texto Lorem Ipsum
\usepackage{xcolor} %Permite el uso de colores
\usepackage{graphicx} %Manejo de imágenes
\usepackage{hyperref} %Referencias cruzadas
\usepackage{booktabs} %Uso de tablas con estilo de libro
\usepackage{tabularx} %Uso de tablas
\usepackage{float} %Uso de entornos flotantes como figure y table
\usepackage{amsmath} %Uso de expresiones matemáticas

%opening
\title{Este es un título}
\author{Este es un autor}
\date{\today} %\today coloca la fecha del día de hoy, pero puedes escribir cualquier fecha

%%----------------------------------%%
%%-Comienza el cuerpo del documento-%%
%%----------------------------------%%

\begin{document}

\maketitle

\begin{abstract}
 Este es un resumen. Si no necesitas un resumen, solo puedes quitar esta sección.
\end{abstract}

\section{Esta es una sección}
\lipsum[1]

\subsection{Esta es una subsección}
\lipsum[2]

\subsubsection{Esta es una subsubsección}
\lipsum[3]

\section{Esta es otra sección}
\lipsum[4]

\section{Estilos de fuente}

Puedo escribir texto {\tiny pequeñitito}, {\footnotesize pequeñito}, {\small pequeñito}, normal, {\large grande}, {\Large más grande} y {\Huge enorme}.\\

También puedo usar \textbf{negritas}, \textit{cursivas}, \underline{subrayado} y \textcolor{red}{cambiar el color de la fuente}. Lo último solo puedo hacerlo si en el preámbulo agrego el paquete \textbf{xcolor}.

\section{Listas}
También puedo escribir listas con viñetas:
\begin{itemize}
	\item Un elemento
	\item Otro elemento
	\item Otro elemento más
\end{itemize}

Además, puedo escribir listas enumeradas
\begin{enumerate}
	\item Un elemento
	\item Otro elemento
	\item Otro elemento más
\end{enumerate}

Puedo anidar listas
\begin{enumerate}
	\item Este es el primer paso que se compone de los dos siguientes
	\begin{enumerate}
		\item Componente 1
		\item Componente 2
	\end{enumerate}
	\item Este es el segundo paso
\end{enumerate}

\section{Figuras}
La Figura \ref{logo} muestra el logotipo de la Universidad de Monterrey.

\begin{figure}[h] \centering 
	\includegraphics[width=0.5\textwidth]{logo_udem}
	\caption{Logotipo de la Universidad de Monterrey}
	\label{logo}
\end{figure}

La Figura \ref{escudo} no aparece inmediatamente después de este texto porque no agregué [h] en el entorno \textbf{figure}. \LaTeX \: acomodará la figura en donde considere apropiado.
\begin{figure} \centering 
	\includegraphics[width=0.5\textwidth]{escudo_udem}
	\caption{Escudo de la Universidad de Monterrey}
	\label{escudo}
\end{figure}

\section{Tablas}
La Tabla \ref{tabla_rara} tiene resultados extraños.

\begin{table}[H]\centering
	\begin{tabular}{ccccc}
		\toprule		
		Performance metric & EC & RLNI & RMD & RR \\ \midrule
		ONVG & 3.75 & 3.33 & 2.50 & 3.67 \\ 
		H & 0.50 & 0.49 & 0.50 & 0.49 \\ 
		Execution time (s) & 1762.20 & 0.03 & 0.02 & 0.04 \\ \bottomrule
	\end{tabular}
	\caption{Average ONVG, H, and execution time values for instances of Class S}
	\label{tabla_rara}
\end{table}

\section{Ecuaciones}
Puedo escribir ecuaciones en línea $x^2$ o ecuaciones fuera de línea $$x^2$$.\\

También puedo escribir ecuaciones enumeradas y con etiquetas como, por ejemplo, la Ecuación \eqref{integral_definida}.
\begin{equation}\label{integral_definida}
	\int_a^bf(x)dx
\end{equation}

Y es posible escribir modelos con más de una ecuación, como el que se muestra en las Ecuaciones \eqref{primera}--\eqref{ultima}
\begin{eqnarray}
	x^2 \label{primera} \\
	\int_a^bf(x)dx \label{segunda} \\
	x+y \label{ultima}
\end{eqnarray}

\subsection{Ejemplos de ecuaciones}
\begin{eqnarray*}
	\lim\limits_{x\rightarrow -1^{-}}f(x)\\
	(-\infty,-\sqrt{2})\cup (\sqrt{2},\infty)\\
	\sqrt[3]{x}\\
	f(x)=\log_{10}\left(\frac{1+x}{1-x}\right)^{2}\\
	\left(\frac{1}{4},\frac{\pi}{6}\right)\\
	\int_{0}^{\pi}t\cos^{2}tdt\\
	\int \theta \tan^{2}\theta d\theta 
\end{eqnarray*}


\end{document}
