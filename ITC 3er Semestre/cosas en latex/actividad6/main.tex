\documentclass[a4paper,12pt]{article}
\author{}
\date{02/11/2023}
\usepackage[papersize={216mm,330mm},tmargin=20mm,bmargin=20mm,lmargin=20mm,rmargin=20mm]{geometry}
\usepackage[español]{babel}
\usepackage[utf8]{inputenc}
\usepackage{amsmath,amssymb,mathabx}%\for eqref
\usepackage{lscape}
\usepackage{graphicx}
\usepackage[colorinlistoftodos]{todonotes}
\usepackage{fancyhdr}
\usepackage{booktabs}
\usepackage{siunitx}

\graphicspath{{./imagenes/}}
 
\pagestyle{fancy}
\fancyhf{}
\lhead{Universidad de Monterrey}
\chead{\thepage}
\rhead{Facultad de Ciencias Fisico Matemáticas}
\lfoot{Mediciones y Metrología}
\cfoot{Actividad 6 - Dilatación Térmica}
\rfoot{Rolando Rivas Dávalos}

\title{\Large \textbf{Universidad de Monterrey}\\ Facultad de Ciencias Fisico Matemáticas \\ Mediciones y Metrología \\ \includegraphics{imagenes/UdemLogo.png} \\ \vspace{5mm} \LARGE \textbf{Actividad 6 - Dilatación Térmica} \\ \vspace{1cm} \normalsize por \\ \vspace{1cm} \Large \textit{Rolando Rivas 594276}}

\begin{document}
\maketitle
\section*{Introducción}
La termodinámica es la rama de la física que estudia los procesos físicos en forma de calor. El calor es una forma de energía que se encuentra en la materia y tiene propiedades interesantes. 
\\\\
Una de las propiedades del calor es su capacidad para dilatar metales aumentando su energia calorífica (temperatura). Esta aumenta la distacia entre moleculas disminuyendo su densidad y aumentando su volumen. Este fenomeno se tiene en cuenta a la hora de contruir objetos metalicos (especialmente los de gran tamaño) tales como rieles, estos pueden cambiar la longitud del riel dependiendo de su temperatura. 
\\\\
La unidad que modela la capacidad de un objeto unidimensional para cambiar de longitud dependiendo de su temperatura es el coeficiente de dilatación térmica ($\alpha$). En esta actividad se aplicará calor a palos de metal para medir el cambio de longitud, calcular su coeficiente de dilatación térmica y compararlo con su valor teórico real. 
\\
\section*{Resultados}
Se muestran los resultados del experimento en la tabla 1 en la unidades de longitud ($L$), temperatura ($T$), coeficiente de la dilatación térmica calculado y su porcentaje de error comparado con su valor $\alpha$ teórico. 

\begin{table}[h]
\centering
\begin{tabular}{||c c c c c c c c||}
\hline
$L_{i}$ (m) & $\Delta L$ $(\mu m)$ & $\Delta m$ (m) & $T_{i}$ (°C) & $T_{f}$ (°C) & $\Delta T$ (°C) & $\alpha \times 10^{-5} C^{-1}$  & \%error \\ [0.5ex]
\hline\hline
0.18 & 199 & 0.000119 & 20 & 60 & 40 & 1.65278 & 2.78\%  \\ 
0.18 & 137 & 0.000137 & 20 & 60 & 40 & 1.90278 & 0.146\% \\
0.18 & 166 & 0.000166 & 20 & 60 & 40 & 2.30556 & 3.94\%  \\
\hline
\end{tabular}
\caption{Para cada metal (cobre, latón y aluminio respectivamente) se mantuvieron todas sus variables excepto sus cambios de longitud como constantes.}
\end{table}

Para calcular $\alpha$ se utilizó la fórmula de 
\[\alpha = \frac{\Delta T}{\Delta TL_{0}}\]
que fue comparada con los valores $\alpha$ de libro y se obtuvo el error porcentual puesto en la ultima columna de la tabla 1.

\section*{Discuciones}
El coeficiente mencionado $\alpha$ es una forma de medir la dilatación por temperatura de un metal en un espacio unidimensional. Existen sin embargo variaciones de esta medida para áreas superficiales y volumenes porque la realidad opera en tres dimensiones. La utilidad de $\alpha$ viene a ser lo que facil que resulta medir longitudes de una dimensión.
\\\\
Podemos observar en los resultados un cambio significativo de longitud $\Delta L \pm 0.5\mu m$ el cual es diferente en cada metal. Tambien concluimos que nuestro instrumento de medición es preciso y exacto pues calculamos valores de $\alpha$ cercanos a los valores de libro.

\section*{Conclusión}
Confirmamos que el metal se dilata con un aumento en su temperatura. Esto es relevante en la construcción de infraestructura y electronicos, ya que un aumento de tamaño puede afectar sus estructuras adyacentes a estos metales.
\\\\
Al comprobar este fenemeno podemos confirmar la teoría de la termodinámica en la que los gases se expanden con el calor, entre otros fenómenos termodinámicos.
\\\\
\textit{Doy mi palabra que he realizado esta actividad con integridad académica.}

\end{document}
