\documentclass{article}

% Language setting
% Replace `english' with e.g. `spanish' to change the document language
\usepackage[spanish]{babel}

% Set page size and margins
% Replace `letterpaper' with `a4paper' for UK/EU standard size
\usepackage[letterpaper,top=2cm,bottom=2cm,left=3cm,right=3cm,marginparwidth=1.75cm]{geometry}

% Useful packages
\usepackage{amsmath}
\usepackage{graphicx}
\usepackage[colorlinks=true, allcolors=blue]{hyperref}

\title{Evidencia no.1}
\author{Rolando Rivas 594276}


\begin{document}
\begin{figure}
\centering
\includegraphics[width=0.5\linewidth]{UdemLogo.png}
\end{figure}

\maketitle

\begin{abstract}
En esta actividad el alumno utilizará los conocimientos de los temas del parical 1 para aplicarlos en el análisis y solución de un problema aplicado a la ingeniería. Además, plasmará su interpretación y reflexión sobre el trabajo realizado.
\end{abstract}

\section{Introduction}

Dado un caudal por donde pasa un fluido de radio constante $R$ el perfil de la velocidad del fluido a una distancia radial $r$ del centro y una constante $k$ esta dado por \[v(r) = k(R^2-r^2)\] 

por lo tanto el caudal volumentrico $Q$ resultaria de calcular la integral definida de $v(R)$ en terminos de $r$: \\ \[ Q =\int_{a}^{b} k(R^2-r^2) \,dr \] \\

A continuacion se muestra el programa en matlab hecho para hacer el calculo:

\begin{verbatim}
>> syms k R r
>> f(r) = k*(R^2 - r^2)
>> d(r) = int(f)
 
d(r) =
 
(k*r*(3*R^2 - r^2))/3
\end{verbatim}

La funcion resultante seria
\\
\[Q =\frac{kr(3R^2 - r^2)}{3}\]
\\
Puede notar que no 
Doy mi palabra que he realizado esta actividad con integridad academica

\end{document}